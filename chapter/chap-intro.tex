
\chapter{绪论}
\section{研究背景与研究意义}
\par 近年来,随着微电子以及通信技术的不断进步与发展,各类小型化,具有感知、存储、通信和计算能力的设备开始普及与应用。伴随而来的是围绕着以人的需求为中心而展开的各类服务技术逐渐多样化和智能化。在各类服务应用中,一项重要的技术支撑——人体行为识别技术,越来越受到人们的关注。人体行为识别技术是借助于布置在人体周围的感知设备,采集并获取用户行为相关的数据,通过现有的有信号处理方法和模式识别技术,对用户当前行为做出识别和判断,从而为用户提供更加智能化的服务。基于行为识别的应用十分广泛,包括医疗健康、人机交互、智能家居、体育娱乐等诸多方面\scite{lockhart2012applications}。在医疗健康方面,行为识别可以用于检测老年人跌倒或者精神病患者的异常行为举动等,以便于及时发现和救助\scite{avci2010activity}。在人机交互方面,系统根据用户行为判读其需求及意图,及时进行系统控制调整,减少用户主动参与。在智能家居场景中,家庭管理系统根据用户当前行为判断用户需求,然后控制家居设备做出调整,使得服务更加智能化。另外,在体育和娱乐方面行为识别也有着广泛的应用,比如随身运动监测、体感游戏等。

\par 人体行为识别技术从上世纪80年代开始发展\scite{surveyOnSensors},最初主要局限在计算机视觉领域,而近些年随着传感器等硬件设备的发展,越来越多的研究者开始关注基于可穿戴传感器的行为识别技术。到目前为止,根据感知设备的不同,行为识别技术可以分为两大类,分别是基于计算机视觉的行为识别和基于可穿戴传感器网络的行为识别。前者主要利用摄像头作为感知设备,通过处理拍摄得到的图像序列或视频片段识别图像中人的行为\scite{surveyOnVision}。图像序列包含关于人体姿势与行为的大量细节信息且易于为人所理解,而且计算机视觉相关技术发展也日臻成熟与完善,因此较多的研究者借助于计算机视觉技术,通过图像处理的方法提取有关人体行为的特征,进而实现人体行为识别。在这一研究领域也取得了较多的研究成果,获得了很高的识别准确率。但是这一技术是利用摄像头等设备采集图像信息,它有其自身固有的局限性。首先摄像头等设备通常被布设在一定区域范围内,所以无法长期获取用户的图像信息,不利于对用户长期进行行为监测与识别。其次是个人隐私问题,通过摄像头等设备拍摄的图像包含大量有关用户的细节信息,而用户通常不愿意过多地暴露个人信息,尤其是在智能家居等隐私敏感性场景下,这也给安全问题带来更大的挑战。最后是复杂性问题,图像处理一般都需要较为复杂的计算,这就给在进行行为识别的移动设备等硬件提出了更高的要求,因此需要配置较为复杂的系统,成本较高,对行为识别的应用与普及带来了负面的影响。相比较而言,基于可穿戴传感器的行为识别技术\scite{sensorBased}则通过布设在人体身上的传感器采集人体行为有关的数据,一般是使用加速度传感器采集加速度数据,然后使用信号处理的方法和模式识别技术对人体行为进行识别与决策。因此基于可穿戴传感器的行为识别技术具有随时随地感知与识别并且隐私侵入小等优点,便于长期的行为监测。但是,基于可穿戴传感器就必须使用户在多个特定位置佩戴一定数量的传感器\scite{kozina2013efficient},必然造成一定的用户侵入性,使得用户体验较差,不利于行为识别技术的应用与普及。另外,多个布置在人体周围的传感器用户采集数据,也为传感器的能耗以及传感器与中心节点的数据通信问题带来了很大的挑战。

\par 鉴于上述技术存在的问题,近几年一些人开始提出使用智能手机进行行为识别的方法\scite{lane2010survey}。随着嵌入式技术的发展和硬件工艺水平的不断提高,传感器模块体积越来越小。各个硬件厂商为满足用户不断增加的需求,将越来越多的传感器设备集成到智能手机中。目前一般的智能手机通常都会集成有加速度传感器、磁场传感器、陀螺仪、温度传感器、GPS、压力传感器等硬件设备\scite{smartphoneSensor}。除此以为,近些年只能手机的存储和计算能力也取得的很大的进步与发展。智能手机逐渐成为集感知、数据处理以及无线通讯为一体的智能化综合平台,拥有着强大的计算能力和数据通讯能力\scite{rashvand2015smartphone}。因此,基于智能手机就可以实现从底层的数据采集到中间层的数据处理,再到上层的应用开发和服务提供以及云存储等功能。另一方面,随着智能手机的普及,现在几乎每个人都会配有一台只能手机,另外由于其与用户的密切关系,智能手机内置传感器可以更加方便且长时间地采集人体行为相关数据,而且没有额外的传感器负担和侵入性\scite{berchtold2010actiserv}。除此之外基于智能手机的行为识别还拥有诸多其他的优点。首先,由于感知设备和数据处理设备都是集成在智能手机内部,因此我们不再需要考虑传感器节点与中心节点的通信问题,以及因为传感器能耗和不易充电引起的传感器的生命周期问题\scite{gyHorbiro2009activity}。其次智能手机内部不但集成了基于可穿戴传感器的行为识别技术中经常用到的加速度传感器,而且还集成有其他运动传感器如磁场传感器、陀螺仪等可用于行为识别。最后,智能手机不断提升的存储和计算能力也很大程度上提高了识别的实时性。因此,基于智能手机的行为识别使用更加灵活方便,具有用户入侵性小,便于长期监测等优点,有利于行为识别技术的应用和普及。
\par 虽然基于智能手机的行为识别相比对于可穿戴传感器,使用更加方便灵活且侵入性小,但是随之带来了一些其他的问题和挑战。首先,不同与可穿戴传感器,智能手机相对于人体的方向并非固定的,因此运动类型的传感器如加速度传感器等采集到的各轴向的传感器数据会随着方向的变化而变化,这给后续的数据处理以及行为识别造成很大的影响。其次,智能手机并不是像可穿戴传感器那样固定或绑定在人体上,它所处的位置也会不断发生变化,运动类型传感器采集的数据也会发生相应变化,尤其是对于动态下的行为,即使在相同行为下,传感器所采集数据也会具有很大的不同。这些也将会对最后的行为识别过程造成很大的影响\scite{siirtola2012recognizing}。最后还有一个不能忽略的问题,相比于多个可穿戴传感器,智能手机在一段时间内仅能够采集人体单个位置的数据,而可穿戴传感器则可以通过部署在人体多个位置的传感器采集更加有效的数据,数据的不充分性也为后续的行为识别提出更大的挑战\scite{kwapisz2011activity}。以上是在如何提高识别结果上存在的问题与挑战,但是在移动环境这个特殊的系统条件下,能量消耗也是不得不考虑的问题。随着智能手机不断发展,移动设备上的应用也越来越广泛,人们使用智能手机也愈加频繁,而想要通过智能手机在后台对人体的行为进行长期监测,必须考虑如何降低在数据采集和行为识别过程中的能耗\scite{benbasat2007framework},从而减少因为行为识别对用户的影响,提高用户体验。
\par 通过以上介绍可以发现,实现使用智能手机对人体行为进行识别与监测,首先是需要克服或者降低因为手机方向和位置变化对行为识别造成的影响,其次在保证一定的识别准确率的同时,努力降低在整个过程中的能量消耗,才可以使得这一技术为用户所接受,便于基于行为识别的各项应用的推广与普及。因此,本文首先针对基于智能手机的行为识别的研究,重点解决方向变化和位置变化问题,期望降低它们对行为识别造成的影响。其次,在保证一定准确率的同时,研究如何调整数据采集和识别过程中的一些变量,从而达到尽可能降低能耗的目的。

\section{国内外研究现状与发展趋势}
\par 由于人体行为识别的应用背景极其广泛,并且随着通信技术、计算机技术以及模式识别等领域的不断发展与完善,利用感知设备获取人体环境信息对人体行为感知和识别的研究也越来越受到研究者的广泛关注。人体行为识别在上世纪80年代就已经被提出,而在刚开始人们对它的研究主要集中在计算机视觉领域,即通过基础设施中的摄像头等设备捕获关于人体行为的图像,进而通过图像处理方法提取特征对行为分类。一般来说,这种方法仅可应用于部分可拍摄到的区域\scite{oneSensorbased},应用场景受限。此外还存在侵犯隐私,计算设备复杂,成本较高等缺点\scite{reviewOnSensorbased}。近些年,随着微电子技术、集成电路、传感器技术等领域的不断进步,一些人开始关注于基于可穿戴传感器的行为识别方案\scite{surveyOnSensorBased}。与基于计算机视觉的行为识别相比,使用可穿戴传感器较少依赖周围环境,同时具有低功耗、低成本、不侵犯隐私信息等优点。但是用户需要在身体的一些特定位置布设一些传感器,因此增加了用户的负担,存在用户侵入性,用户体验较差。另外基于可穿戴传感器的行为识别还需要考虑传感器与中心节点之间的通信以及传感器的能耗等问题。而在最近几年,智能手机逐渐走进人们的日常生活并且不断普及,由于智能手机通常都会集成加速度传感器等感知设备,并且具备较强的存储和计算能力\scite{surveyOnMobileSensing},因此研究者开始提出仅使用智能手机对人体行为进行监测和识别。基于智能手机的行为识别具有方便灵活的特点,并且对用户没有额外的负担,便于行为识别技术应用的推广。但是,由于智能手机所处环境的不可控性,如手机的方向与位置的变化等,这一项技术依然存在很多的问题和挑战。此外,随着智能设备不断进步,其存储和计算能力不断提升,而且在智能手机内部集成越来越多种类的传感器,这些也给基于智能手机的行为识别研究提供了更强大的平台,人们也开始提出很多在可穿戴传感器中难以实现的方法用于机遇智能手机的行为识别技术。另外一方面,由于移动设备的能量都是受限的,而通过智能手机采集人体行为相关数据并做行为识别又是需要长期执行的任务,因此部分研究者开始关注如何调整策略降低在识别过程中的能耗问题。本节的剩余部分将从基于智能手机行为识别的研究和降低识别过程中能耗的策略研究两个方面介绍目前国内外的研究状况。

\subsection{基于智能手机的行为识别研究}
\par 智能手机内部集成的传感器包括很多种类型,按照其功能可以将其分为运动传感器和环境传感器。对于基于智能手机的行为识别研究可以根据使用传感器的种类不同分成两类。第一类是借鉴基于可穿戴传感器的行为识别中的方法,使用加速度传感器监测用户的运动信息,在使用智能手机采集数据时还可以使用磁场传感器、陀螺仪等获取其他的运动信息,辅助行为识别与决策。第二类是在使用运动传感器做行为识别的同时还研究了其他环境传感器所采集的关于用户的环境信息对行为识别的帮助。
\par 对于智能手机中的运动传感器,Muhammad Shoaid等人在\scite{diffSensors}中通过对比实验研究了除加速度传感器以外,陀螺仪和磁力计对行为识别的影响,实验分别对不同位置和不同行为分别做了对比实验,结果表明它们对不同位置的不同行为所起到的作用会有所不同,且区别较大,比如陀螺仪在区分上下楼和行走的过程中则具备较好效果,对其他行为区分度则不大。在一些研究中,继续沿用基于可穿戴传感器的行为识别方法,而将传感器替换为智能手机,进行人体行为识别\scite{fixPositionandOrientation1}\scite{fixPositionandOrientation2}。他们在实验室内通过将智能手机绑定在裤子口袋位置,并且规定了手机相对于人体的方向,即通过假定假定手机的位置和方向固定不变,重点研究三轴方向中特定方向上特征对行为识别的作用。但是这种假设过于严格与实际相差较大,不利于行为识别应用的推广。为此一些研究者重点研究了智能手机方向和位置变化对行为识别的影响,并提出了一些解决方案。
\par 对于方向的变化问题,S. Thiemjarus 在\scite{orientationProblem}中通过设置智能手机的方向,研究方向对加速度等传感器数据的影响。为降低手机方向变化问题对行为识别的影响,一方面部分研究者通过使用方向独立的数据提取特征,M.B.Rasheed 等人在\scite{fixOrientationData}中使用传感器的三轴方向数据的信号模值向量最后的数据用于特征提取,从而避免方向变化带来的影响;另一方面则是利用旋转矩阵的方法作用于数据向量,进而减少方向变化带来的干扰。Anjum等人和S. Thiemjarus 等人分别在\scite{orientationTransation1}\scite{orientationTransation2}中提出使用旋转矩阵对传感器数据做变化,从而降低方向变化对行为识别的干扰。他们分别提出利用各轴向数据间的相关系数矩阵进行特征值分解以及基于投影的方法,计算得出旋转矩阵进而对数据进行旋转变换,减弱因方向变化对数据产生的干扰。虽然基于旋转矩阵可以一定程度上解决智能手机方向变化的问题,但是计算得到的旋转矩阵是手机坐标相对于地球坐标的角度,旋转以后的数据无法反映出用户身体相对于地球坐标系的变化,在一些行为的识别过程中丢失了方向信息。
\par 对于位置的变化问题,一方面是使用位置独立的特征或分类方法,Changhai Wang等人在\scite{frequencyForPosition}中提出频域的特征受到位置变化的影响较小,使用频域特征时,在相同行为下,手机被放置在不同位置行为识别正确率较高。但是频域的特征只能适用于动态的行为,对于静止的行为则无能为力了。另一方面则是针对每个位置使用特定位置所采集的数据去训练不同的分类模型,但是在识别阶段这需要位置分类器首先识别其位置信息,然后使用不同的特定位置分类模型对行为进行识别分类。D. Kelly 等人在\scite{positionClassifier} 提出在行走状态下使用位置分类器检测智能手机在人体上的位置,但是这个位置分类器仅能够在人体行走状态下才能工作。而且在\scite{positionClassifierProblem}中,作者通过对比实验表明,位置独立的分类模型总体效果并没有优于所有位置的同一分类模型,这是因为位置分类器的产生的错误分类对最终结果产生较大的影响。最后一些研究者则关注了单一位置采集数据造成数据集不能充分表征行为特征的问题。在\scite{otherSensorCompensate}中,作者研究表明多种类型的传感器数据可以补偿因为单一位置数据的不足。在\scite{othersensorInstance1}\scite{othersensorInstance2}等文献中,他们通过融合陀螺仪等其他智能手机中常见运动传感器数据进行行为识别,收到较好的效果。
\par 除使用运动传感器以外,部分研究尝试使用智能手机内置的其他类型的情景感知传感器,通过检测用户周围的情景信息辅助行为检测。A.E.Halabi 等人在\scite{pressureSensor}中研究通过结合使用手机内置压力传感器检测周围大气压力的变化用于区分上楼和下楼两种行为。R.D.Das 等人在\scite{gps}中则利用GPS信息辅助行为识别的最终决策。在他们的研究中都一定程度上提高了识别准确率,但是由于这类传感器并没有集成在所有的智能手机中,所以使用这类传感器不利于行为识别技术的广泛应用。

\subsection{降低行为识别过程中能耗的策略研究}
\par 作为移动端的设备,智能手机控制应用的能耗是十分有必要考虑的问题。对于降低行为识别过程中能耗问题,研究者主要针对于研究识别准确率与能耗之间的权衡策略。这部分的研究内容包括两个主要的问题,一是建立准确率和能耗模型并且权衡二者作出最佳策略选择,即通过研究准确率和能耗分别与哪些变化因素相关,以及他们之间的关系从而选择最佳策略;第二个问题是在识别过程中如何调整识别策略,从而在保证一定识别准确率的同时尽可能降低能耗\scite{zappi2008activity}。
\par 对于准确率和能耗的影响因素,V.Q. Viet等人和Z.X. Yan等人分别在\scite{fixPositionandOrientation2}\scite{modelVarialb}考虑了传感器的采样率和用于分类的特征集对准确率和能耗的影响。但是他们都是通过采样率和特征集数值对的形式建立离散模型,研究他们对准确率和能耗的影响,并没有建立连续模型关系。在\scite{sensorSet}中,作者考虑了选择最佳传感器集合以及传感器采集数据的周期,在本文中重点研究了二者对准确率和能耗的影响。D. Gordon等人则在\scite{sensorSetAndmodel}同样考虑了传感器的选择,从而产生不同的特征集合,研究它们对准确率和能耗的影响,同时作者通过对比准确率,依照准确率的下降定义行为对特征的依赖程度,进而简历准确率与特征集的函数模型。N.H. Viet等人在\scite{precionModel}则时通过定义信息质量建立识别准确率与传感器的选择之间的函数模型,同时建立了各类消耗模型,从而通过求解目标函数的最优解问题做出最佳权衡决策。但是这些两篇文献都没有考虑准确率和能耗与采样率、采样周期等因素之间的关系模型。
\par 通过研究识别准确率和能耗与一些变化因素之间的函数关系,可以为每一类行为权衡两个因变量选择最佳的识别策略,下一个问题就是如何在识别过程中做出动态的策略调整。在\scite{fixPositionandOrientation2}\scite{modelVarialb}中时提出一种连续帧决策的方法,并引入了置信度的概念,进而在识别过程中根据前一个窗口时间内的识别结果决定当前时刻的策略。O. Yurur等人在\scite{adapteStratery}中则是行为识别与降低能耗在一起考虑,提出使用基于离散时间的非齐次隐半马尔科夫模型进行人体行为识别,同时提出使用约束马尔科夫决策过程和反馈机制去权衡识别准确率和能耗,从而动态地在决策过程中做出最佳策略选择。
\par 虽然之前的研究者对如何降低识别能耗作出了部分的研究工作,但是他们的方法都是通过调节离散的采样率等识别策略权衡识别准确率和识别能耗,并没有提出系统的研究方法,不能很好地应用到不同的行为集合的识别应用中。
\section{研究内容与贡献}
\par 基于上述利用手机进行行为识别目前所存在的问题,本文的主要研究内容包括以下三个方面:首先提出一种基于智能手机的两层多策略的行为识别方案框架,然后是研究降低行为识别过程中能耗的策略动态调整方法,最后在智能手机终端实现行为识别应用,验证方案框架与调整方法的实用性和有效性。

\textbf{1.基于智能手机的两层多策略的行为识别方案框架
}


\par 对于基于智能手机的行为识别过程中存在的手机位置和方向变化问题,本文首先提出使用一系列与数据自相关函数相关的特征以降低手机方向变化带来的影响。其次提出两层多策略的行为识别方案框架应对手机位置变化的问题。在该方案框架中,所有行为首先根据行为间的相似度被分为若干组,然后在第二层中,根据不同组的行为特点选择相应的最佳策略进行分类,并对动态行为引入位置分类器降低位置变化对识别结果的干扰。
\par 该部分研究点的具体研究内容如下:

\begin{itemize}
\item 特征提取:首先通过智能手机内置的加速度传感器,磁力计和陀螺仪采集人体行为数据,然后对数据做预处理并通过加窗的方法进行分段,最后可以对窗口内的数据实例提取特征。在我们框架中采用了时域、频域和自相关函数相关的特征,其中频域特征可以有效降低位置敏感性,而在本文中又提出若干自相关函数的特征,与方向无关,可以有效降低方向变化对行为识别的干扰。
\item 分组模型:在分类框架的第一层中,所有行为首先根据行为相似度分为若干组,如静态行为,慢速动态行为和快速动态行为等,然后以分组后的数据作为训练集训练分组模型,该分组模型用于框架的第一层将一个窗口内的行为实例分到某一个组。对于已经分组的行为,则可以在方案的第二层中根据不同组行为的不同特点采用不同的策略将组内的行为进一步分类,获取最终结果。
\item 静态行为分类策略:对于静态行为,比如坐和站,很难通过运动类型传感器的数据分析二者之间的区别,因此在本文中提出通过识别静态行为转变过程中的过渡态行为,如站起和坐下,进而判断当前时刻的静止行为状态。
\item 动态行为分类策略:对于动态行为,由于位置的变化对运动类型传感器所采集的运动数据模式造成很大的影响,因此本文针对动态行为引入位置分类器,对于动态行为针对不同位置训练特定位置的分类模型。对于动态行为引入位置分类器,一方面可以在运动状态下为位置识别提供较好的条件,提高位置识别准确率,另一方面又可以根据位置信息选择特定位置分类器,降低位置变化的干扰。
\end{itemize}


\textbf{2.降低行为识别过程中能耗的策略动态调整方法
}


\par 降低识别过程中能耗的基本思想就是通过调节一些影响因素,权衡识别准确率与能耗之间的关系,进而选择最佳的策略用于行为识别。在本文中,首先通过研究准确率和能耗与影响因素的关系,建立数学模型。然后针对每一类行为构建目标函数权衡准确率与能耗的关系,通过求解最优化问题为每种行为选择最佳的识别策略。最后制定一套自适应的动态调整方法,使得最终的识别过程在保证一定准确率的基础之上进可以降低能耗。
\begin{itemize}
\item 准确率与能耗模型:在本文中主要考虑的影响因素主要包括传感器的采样率和在一个窗口时间的采样时间比例以及选择用于分类的特征集,即是否使用频域特征,是否使用自相关函数的特征。准确率和能耗与这些影响因素的关系则是首先建立数学模型,然后通过实验测量数据进行拟合或者计算模型中的参数,从而获得二者与上述影响因素变量之间的模型关系。
\item 最优化问题:首先对于每一类行为都可以计算其相应准确率模型以及适用于所有行为的能耗模型,然后当前电量可以作为准确率和能耗的权衡系数,从而构建出目标函数,最后求解该目标函数的最优值而获得适用于当前电量以及该行为的最佳策略,包括应当选择的采样率,采样时间比例,是否使用频域以及自相关函数特征等。
\item 动态调整方法:根据计算获得的每种行为的最佳策略,可以根据识别过程的实时结果作出动态调整。此部分的关键问题在于判定行为的转变,本文通过结合行为马尔科夫模型的转换概率和当前的行为识别结果,综合考虑判定行为转换结果,进而对识别策略作出动态调整,达到降低能耗的目的。
\end{itemize}

\textbf{3.人体行为识别应用
}
\par 本文最后将在运行Android系统的智能手机上实现一款人体行为识别应用,该应用中实现了我们的两层多策略的识别框架和降低能耗的最佳策略动态调整方法,并在实际的应用中统计识别结果以及能量消耗,从而验证本文方法的实用性与有效性。

\section{本文结构与安排}
\par 本论文总共分为六章,各个章节的内容安排如下:
\par 第一章是绪论,在本章中首先介绍了课题的研究背景与研究意义以及该研究方向的国内外研究现状,然后简要阐述了本文的主要研究内容和贡献,最后简要介绍本文的结构与安排。

\par 第二章主要介绍了基于智能手机行为识别的相关背景与基础知识。首先简要介绍了行为识别的基本流程与框架以及需要解决的问题和存在的挑战,然后介绍了目前智能手机中常见的内置传感器以及其他可用于行为识别的感知设备,最后深入地介绍了行为识别,特别是基于智能手机的行为识别中所用到的特征集和分类模型。

\par 第三章主要介绍了本文提出的两层多策略的行为识别方案框架。首先概述了该方案框架的整体结构和识别流程,然后深入地介绍了框架中的各部分包括数据采集和特征提取,分组模型,静态行为分类策略以及动态行为分类策略,最后给出了实验数据的识别结果以及与其他识别方法的对比实验结果。

\par 第四章主要介绍了降低行为识别过程中能耗的动态策略调整方法。首先简要阐述了降低能耗的基本思想以及为每类行为计算最佳策略的基本方法,然后详细地介绍了该部分的各项内容,包括数学模型,最优化问题与求解以及动态的策略调整方法,最后给出使用该调整方法后的识别结果以及对比实验结果。

\par 第五章主要介绍了本文中实现的人体行为识别应用,包括程序的主要结构框架,各部分的主要功能,主要界面以及操作方式等。

\par 第六章首先对本文的主要内容作出了总结,然后对基于智能手机行为识别的相关研究做了展望。
