\chapter{总结与展望}

\section{工作总结}
\par 基于智能手机进行行为识别,使用方便灵活,便于对用户行为进行长期监测,有利于行为识别应用的推广。但是智能手机存在着方向和位置变化的问题,为此本文提出了一种基于智能手机的两层多策略的行为识别框架。首先将所有行为根据相似度分为若干组,然后针对不同组的行为特点采样对应策略对其进一步分类,最终完成行为分类。同时,为了在保证一定识别准确率的前提下,尽可能降低识别过程中的能量损耗,本文提出了一种最佳策略的动态调整方案,根据手机当前电量以及识别结果,动态调整数据采集和特征提取策略,以达到降低识别过程中能耗的目标。最后为了验证本文所提出方案的有效性和实用性,本文实现了一款可运行于Android操作系统之上的行为识别手机应用,同时该应用可以与远端服务器完成数据通信和交互,形成一套相对较为完善的行为识别系统。
\begin{enumerate}
	\item 为了应对在行为识别过程中智能手机方向和位置的变化问题,本文首先提出了使用自相关函数的特征,通过针对传感器数据各轴向上的自相关函数之和提取特征,从而有效降低因为手机方向变化对行为识别的影响。其次,本文提出了两层多策略的行为识别框架,可将行为分组后再根据不同的行为特点进一步分类,使得特征选择更有针对性。具体地,对于静态行为本文使用识别过渡状态行为的方法判别静态行为,对于动态行为则是通过位置分类器获取位置信息后,使用特定位置数据训练的分类器做最终的行为分类,有效降低了位置对行为识别的影响。
	\item 本文提出了最佳策略的动态调整方法,可以在保证一定识别准确率的前提下尽可能降低识别过程中的能量损耗。为了实现这一目标,本文首先研究准确率和能耗两项指标与若干影响因素的关系,并选定其中较为重要的四个影响因素(采样率,采样时间,是否使用频域特征,是否使用自相关函数特征)作为变量建立数学模型,然后通过实验测量和计算的方法求解数学模型参数,获得两项指标关于这四个变量的函数关系。通过电量作为系数权衡二者关系建立目标函数,进而将求解最佳策略问题转换为求解关于目标函数的最优化问题,并求解后获得每类行为每种电量等级下的最佳策略。最后本文使用一种动态调整的方法,根据手机的当前电量以及识别结果动态调整最佳策略,最终实现降低能耗的目的。
	\item 本文最后介绍了已经实现基于只能手机的行为识别系统,并且从控制层,算法实现,模型层和视图层四个部分详细介绍了手机应用的组成和功能,根据此应用在实际中的识别准确率和能耗说明本文所提出方法的有效性和实用性。
\end{enumerate}	
\section{研究展望}
\par 通过感知人体的运动和环境信息识别用户当前行为,进而为用户提供更为智能化和人性化服务具有十分重要的意义,因此行为识别技术一直受到研究者的广泛关注。本文是从运动类型信息出发,重点为克服位置和方向变化问题,以及降低能耗问题做了深入研究,但是还有一些关于基于智能手机的行为识别本文所没有涉及的问题,可以继续进行深入的研究。主要包括以下几点:
\begin{enumerate}
	\item 日常行为识别:在实际生活中,识别用户的日常行为对于提供智能化服务更有意义,但是日常行为并非只是简单的几类行为,而且行为之间通常是连续的,可能没有明显区分间隔。研究如何定义行为并限定分类的范围,进而识别日常行为会使得行为识别技术更加贴近于实际。
	\item 多人行为:社交网络是一个非常热门的领域,人与人的关系错综复杂却拥有一定的规律性。同时研究多人的行为并理解不同用户行为之间的关系规律,对于行为识别的研究具有很重要的影响。
	\item 其他类型传感器:随着智能手机的不断发展,将会有越来越多的新型传感器集成到智能手机中,可以通过此类传感器获取用户更加丰富的行为信息。与此同时,已经存在但还没有完全普及的传感器如压力传感器,温度传感器等,也将会被集成到越来越多的智能手机中。因此研究使用此类环境信息或其他类型信息的传感器,而不再局限到运动类型传感器对于基于智能手机的行为识别技术将会有很大的帮助。
	\item 传感器的选择:当可选择传感器的范围较为广泛时,就有必要研究选择传感器集合的问题,可研究如何选择合适的传感器集合以适应当前的行为识别,从而在保证准确率的同时减低识别能耗。
\end{enumerate}