
\begin{thanks}

\par 时光荏苒,岁月如梭,三年紧张而充实的研究生学习生活已经接近尾声,回顾在中国科技大学的这段宝贵的时光,我的内心充满无限的感怀与眷恋。在毕业论文即将完成之际,我要向这三年里所有关心、帮助和鼓励的人致以最真诚的感谢!感谢你们的支持让我在研究生学习阶段在学习、科研和生活方面都受益良多。
\par 首先,感谢我的导师刘斌副教授的谆谆教诲和悉心指导,刘老师耐心地教导我做科研的方法并毫无保留地与我分享他丰富的生活与科研经验,对我的个人成长以及科研工作都具有极大的帮助。在过去的三年,刘老师对我的学习和科研生活倾注了大量精力和心血,为我提供了大量的学习与锻炼的机会,并在我面临生活和科研问题时给予我很大的帮助和鼓励。刘老师知识渊博,治学严谨,并且对科研问题的把我十分到位,总是可以一针见血地指出我科研工作中存在的问题并给予切实可行的解决意见。此外,刘老师深厚而扎实的学识功底、严谨的研究态度、敏锐的洞察力以及宽广的视野与胸怀都使我受益匪浅。在此向刘老师致以最诚挚的谢意!
\par 同时,十分感谢研究生期间其他老师、同学和朋友们的关心和帮助。在此要特别感谢俞能海教授对我的指导和教育。俞老师不仅在科研上给予我以极大的帮助和鼓励,并且在思想觉悟、认知高度和人生职业规划上使我有了全新的认识。此外,还要感谢6系的各位老师们的辛勤教诲,他们使我拥有更加扎实的基础知识以及更加广阔的知识领域认知和理解。同时,还要感谢实验室的兄弟姐妹,他们有已经毕业的师兄师姐包括王明琛、何磊、陈珏、危江月、丰慧、余宿城、罗金国、尤勇、谢宇、吴文超、刘可佳,还有在读的包括邹磊、刘志强、赵孝松、罗浩、云凡、蒋浩然、管程杰、殷国军、金红星、李航、陈胜以及大实验室的其他同学,在此还要特别感谢邹磊、余宿城、刘志强和金红星同学在本文的实验部分给予的支持和帮助。此外,还要感谢306的室友及好友廖鹏、杨卓煦、林阳,谢谢你们的无私帮助。
\par 最后,需要特别感谢的是我的家人。他们一如既往地默默奉献,一直鼓励和支持着我,是我求学道路上最为坚强的后盾。感谢您们的默默付出,感谢您们的支持与帮助,您们对我的无私的爱以及为我营造的宽松的学习环境和氛围使我一生最为宝贵也是最需要去珍惜的财富。
\par 感谢科大,感谢一路走来的兄弟姐妹们,在最宝贵的青春年华里,感激与你们的相遇,感谢你们伴随我的成长。最后为每一位曾经帮助和支持过我的老师、同学和朋友致以最诚挚的谢意!

\vskip 10pt

\begin{flushright}
~~~~郭乾~~~~

\today

\end{flushright}
\end{thanks}
